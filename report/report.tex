\documentclass[a4paper,12pt]{article}
\usepackage{graphicx} % For including images
\usepackage{amsmath}  % For mathematical formulas
\usepackage{lipsum}   % For dummy text
\usepackage{fancyhdr} % For header and footer customization
\usepackage{polyglossia} % For multilingual support
\usepackage{geometry} % For page layout
\usepackage{placeins}
\usepackage[colorlinks=true, citecolor=blue]{hyperref}
\usepackage[backend=biber]{biblatex}

% Set the languages
\setmainlanguage{english}
\setotherlanguage{greek}

% Define fonts
\setmainfont{Linux Libertine}
% \newfontfamily\greekfont{Linux Libertine}
\setmonofont{UbuntuMono Nerd Font Mono}

% Page settings
\geometry{margin=1in}

\begin{document}
\begin{otherlanguage}{greek}

% Cover Page
\begin{titlepage}
    	\centering
    	\vspace*{2cm}

    	\Huge
    	\textbf{Προχωρημένα Θέματα Βάσεων Δεδομένων}

    	\vspace{0.5cm}
    	\LARGE
    	Εξαμηνιαία Εργασία \\
	\Large
    	Σχολή Ηλεκτρολόγων Μηχανικών και Μηχανικών Υπολογιστών, ΕΜΠ

    	\vspace{1.5cm}
    	\includegraphics[width=0.5\textwidth]{ntua.png}

    	\vfill

   	\Large
	\textbf{Ομάδα 7} \\
	Οικονόμου Νικόλαος (03120014) \\
	Ραφτόπουλος Μιχαήλ (03120114) \\

	\vfill

	\normalsize
    	Ιανουάριος 2025

    	\vspace{0.8cm}
\end{titlepage}

\newpage

% Table of Contents
% (Maybe we should delete that)
\tableofcontents
\newpage

% Main Content

\section*{Query 1}
	\addcontentsline{toc}{section}{Query 1}
	\FloatBarrier
	Εκτελέστηκε το query 1, τόσο με το DataSet API, όσο και με το RDD API του Spark.
	Συγκρίνοντας τον χρόνο των δύο υλοποιήσεων, η υλοποίηση με DataSet API
	αποδείχθηκε ταχύτερη\footnote{Σε αυτό το σημείο να αναφερθεί
	ότι στις διάφορες δοκιμές του χρόνου
	εκτέλεσης των queries τα αποτελέσματα παρουσιάζαν πολύ μεγάλη διακύμανση, που
	πιθανώς οφείλεται στο cloud περιβάλλον εκτέλεσης. Οι σχετικοί χρόνοι όμως
	ανάμεσα στις υλοποιήσεις παρέμεναν σταθεροί. Για το λόγο αυτό, στο μεγαλύτερο
	μέρος της εργασίας αναφερόμαστε σε σχετικούς χρόνόυς και όχι απόλυτα νούμερα.}.
	Το αποτέλεσμα αυτό είναι αναμενόμενο, καθώς το DataSet API αποτελεί μια
	υψηλωτερη αφαιρετικά διεπαφή, με πολλές βελτιστοποιήσεις να πραγματοποιούνται
	στο υπόβαθρο. Με το RDD API μπορούμε θεωρητικά να πετύχουμε την ίδια
	αποδοτικότητα, αλλά απαιτείται πολλή εμειρία και προσεκτικός χειρισμός.
	% Δημιουργείται νέο RDD για κάθε operation => περισσότερη μνήμη
	\begin{table}[h]
		\centering
		\begin{tabular}{cc}
			Age Group & Count \\
			\hline
			Adults & 121093 \\
			Young Adults & 33605 \\
			Children & 15928 \\
			Elderly & 5985
		\end{tabular}
		\caption{Αποτελέσματα query 1.}
	\end{table}
	\FloatBarrier

\section*{Query 2}
	\addcontentsline{toc}{section}{Query 2}
	\FloatBarrier
	\par{(α)} Υπολογίστηκαν για κάθε έτος τα 3 Αστυνομικά Τμήματα με το υψηλότερο
	ποσοστό κλεισμένων υποθέσεων και ταξινομήθηκαν ανά έτος και ανά ποσοστό.
	Χρησιμοποιήθηκαν δύο διαφορετικά APIs του Spark: το DataFrame API και το SQL
	API. Μετά από αρκετές επαναλήψεις, η υλοποίσηση με SQL API αποδείχθηκε 
	ταχύτερη από το αυτήν με DataFrame API. Θεωρητικά δε θα αναμέναμε ουσιαστική
	διαφορά μεταξύ των δύο, καθώς αποτελούν απλά διαφορετικές διεπαφές του ίδιου
	optimizer. Η απόκλιση λοιπόν των δύο υλοποιήσεων μάλλον οφείλεται στον τρόπο
	που αυτές είναι γραμμένες, με τον κώδικα σε DataFrames να οδηγεί σε περισσότερα
	operations.
	\par{(β)} Σε αυτό το σημείο, έγινε η μετατροπή των δεδομένων εισόδου από 
	\texttt{.csv} σε \texttt{.parquet}. Εκτελέστηκε η ίδια υλοποίηση SQL,
	χρησιμοπιώντας το δεύτερο format. Η υλοποίηση με τα δεδομένα σε
	\texttt{.parquet} ήταν ταχύτερη. Αυτό είναι αναμενόμενο, καθώς το 
	\texttt{.parquet} αποτελεί έναν τύπο αρχείου βελτιστοποιημένο για κατανεμημένα
	filesystems.
	\begin{table}[h]
		\centering
		\begin{tabular}{cccc}
			year & AREA NAME & closed\_rate & \# \\
			\hline
			2010 & Rampart & 32.84713448949121 & 1 \\
			2010 & Olympic & 31.515289821999087 & 2 \\
			2010 & Harbor & 29.36028339237341 & 3 \\
			2011 & Olympic & 35.0400600901352 & 1 \\
			2011 & Rampart & 32.496447181430604 & 2 \\
			...
		\end{tabular}
		\caption{Aποτελέσματα query 2 (φαίνονται μόνο οι 
		πρώτες γραμμές).}
	\end{table}
	\FloatBarrier

\section*{Query 3}
	\addcontentsline{toc}{section}{Query 3}
	\FloatBarrier
	\lipsum[3] % Placeholder text. Replace with your content.
	\begin{table}[h]
		\centering
		\begin{tabular}{ccccc}
			COMM & ... & Median Income Per Person & ... & 
			Crimes Per Person Ration \\
			\hline
			Elysian Park & & 13871.32276 & & 1.08487 \\ 
			Longwood & & 13420.05226 & & 0.73017 \\
			Cadillac-Corning & & 19572.7847 & & 0.66692 \\
			...
		\end{tabular}
		\caption{Aποτελέσματα query 3 (φαίνονται μόνο οι 
		πρώτες γραμμές και επιλεγμένες στήλες).}
	\end{table}
	\FloatBarrier

\section*{Query 4}
	\addcontentsline{toc}{section}{Query 4}
	\FloatBarrier
	\lipsum[4] % Placeholder text. Replace with your content.
	\begin{table}[h]
		\centering
		\begin{tabular}{ccc}
			vict\_descent & total\_victims & ... \\
			\hline
			White & 8429 & \\
			Other & 1125 & \\
			Hispanic/Latin/Mexican & 868 & \\
			Unknown & 651 & \\
			...
		\end{tabular}
		\caption{Aποτελέσματα query 4 για την περίπτωση των
		περιοχών με το υψηλότερο κατά κεφαλήν εισόδημα 
		(φαίνονται μόνο οι 
		πρώτες γραμμές και επιλεγμένες στήλες).}
	\end{table}
	\FloatBarrier

\section*{Query 5}
	\addcontentsline{toc}{section}{Query 5}
	\FloatBarrier
	\par Πραγματοποιήθηκε η υλοποίηση του Query 5 και εκτελέστηκε με τα 3
	ζητούμενα configuration. Για τον υπολογισμό του χρόνου εκτέλεσης, το πρόγραμμα
	επαναλήφθηκε 10 φορές για κάθε configuration και υπολογίστηκε ο μέσος χρόνος για
	κάθε περίπτωση. Παρακάτω φαίνονται τα αποτελέσματα:
	\begin{itemize}
		\item 2 executors $\times$ 4 cores/8GB memory: \textbf{9.82s}
		\item 4 executors $\times$ 2 cores/4GB memory: \textbf{7.60s}
		\item 8 executors $\times$ 1 core/2GB memory: \textbf{7.25s}
	\end{itemize}
	\par Βλέπουμε ότι πολλοί «αδύναμοι» executors αποδίδουν καλύτερα απ' ο, τι λίγοι
	«ισχυροί». Η παρατήρηση αυτή οφείλεται στην υψηλή παραλληλησιμότητα των
	λειτουργιών, που βλέπουν μεγάλο όφελος με την κατανομή σε πολλές διεργασίες.
	\begin{table}[h]
		\centering
		\begin{tabular}{ccc}
			DIVISION & crime\_count & mean\_distance \\
			\hline
			HOLLYWOOD & 213080 & 2.269 \\
			VAN NUYS & 211457 & 3.181 \\
			WILSHIRE & 198150 & 2.921 \\
			SOUTHWEST & 186742 & 2.395 \\
			...
		\end{tabular}
		\caption{Aποτελέσματα query 5 (φαίνονται μόνο οι 
		πρώτες γραμμές).}
	\end{table}
	\FloatBarrier

% References
\FloatBarrier
\printbibliography

\end{otherlanguage}
\end{document}
